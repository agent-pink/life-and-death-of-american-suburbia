\documentclass[8pt]{amsbook}
\usepackage{minitoc}
\usepackage{graphicx}
\usepackage{amsmath}
\usepackage[english]{babel}
\usepackage{lipsum}
\usepackage[letterpaper, top=1in, bottom=1in, left=1in, right=1in]{geometry}
%\usepackage[papersize={4.25in,11in}, top=1in, bottom=1in]{geometry}
%\usepackage[papersize={5.5in,8.5in}, top=1in, bottom=1in]{geometry}
%\usepackage[papersize={4.25in,5.5in}, top=0.75in, bottom=0.75in]{geometry}
\title{The Death And Life Of American Suburbia}
\author{George Rogers}
\begin{document}
\dominitoc
\maketitle
Copyright (c) 2016 George Rogers

The MIT License (MIT)

Permission is hereby granted, free of charge, to any person obtaining a copy
of this software and associated documentation files (the "Software"), to deal
in the Software without restriction, including without limitation the rights
to use, copy, modify, merge, publish, distribute, sublicense, and/or sell
copies of the Software, and to permit persons to whom the Software is
furnished to do so, subject to the following conditions:

The above copyright notice and this permission notice shall be included in all
copies or substantial portions of the Software.

THE SOFTWARE IS PROVIDED "AS IS", WITHOUT WARRANTY OF ANY KIND, EXPRESS OR
IMPLIED, INCLUDING BUT NOT LIMITED TO THE WARRANTIES OF MERCHANTABILITY,
FITNESS FOR A PARTICULAR PURPOSE AND NONINFRINGEMENT. IN NO EVENT SHALL THE
AUTHORS OR COPYRIGHT HOLDERS BE LIABLE FOR ANY CLAIM, DAMAGES OR OTHER
LIABILITY, WHETHER IN AN ACTION OF CONTRACT, TORT OR OTHERWISE, ARISING FROM,
OUT OF OR IN CONNECTION WITH THE SOFTWARE OR THE USE OR OTHER DEALINGS IN THE
SOFTWARE.

\tableofcontents
\chapter{Why Suburbia}
\minitoc
\section{Suburbia Through History}
\section{Suburbia And Progress}
Wealth breeds a want for privacy,
therefore as affluence reaches the mainstream suburbia blooms.
The automobile provides the masses with the best of both worlds:
The urban life when you want it, the country estate when you do not.
The automobile \emph{enabled suburbia}. It did not \emph{force suburbia}.
Therefore going back to the pre automobile city is a pipe dream.

Suburbia is the best way to give the masses the best of both the City and Country lifestyles.
Therefore if your anti-suburb you are anti-progress.
\chapter{Roads And Their Usage}
\minitoc
\section{Residential Streets}
\section{Freeways}
\section{Feeders}
\section{Arterials}
\chapter{Strip Malls}
\minitoc
\section{Relationship With Arterials}
\section{Mixing Rents}
\chapter{Malls And Town Centers}
\minitoc
\section{Malls}
\section{Town Centers}
\end{document}
