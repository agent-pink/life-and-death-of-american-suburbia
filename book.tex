\documentclass{amsbook}
\usepackage{minitoc}
\usepackage{graphicx}
\usepackage{amsmath}
\usepackage[english]{babel}
\usepackage{lipsum}
\usepackage[letterpaper, top=1in, bottom=1in, left=1in, right=1in]{geometry}
%\usepackage[papersize={4.25in,11in}, top=1in, bottom=1in]{geometry}
%\usepackage[papersize={5.5in,8.5in}, top=1in, bottom=1in]{geometry}
%\usepackage[papersize={4.25in,5.5in}, top=0.75in, bottom=0.75in]{geometry}
%\usepackage[a6paper]{geometry}

\title{The Death And Life Of American Suburbia}
\author{George Rogers}
\date{}
\begin{document}
\setcounter{chapter}{-1}
\dominitoc
\maketitle
\input{license}
\tableofcontents

\chapter{Why Suburbia}
\begin{quotation}
    Suburbia is the best way to give the masses the best of both the City and Country lifestyles.
    Therefore if your anti-suburb you are anti-progress.
\end{quotation}
\minitoc

In \cite{rBruegmann05}, Robert Bruegmann explains the hows and whys of suburbanization throughout history.

Wealth breeds a want for privacy,
therefore as affluence reaches the mainstream suburbia blooms.
The automobile provides the masses with the best of both worlds:
The urban life when you want it, the country estate when you do not.
The automobile \emph{enabled suburbia}. It did not \emph{force suburbia}.
Therefore going back to the pre automobile city is a pipe dream.

Suburbia is the best way to give the masses the best of both the City and Country lifestyles.
Therefore if your anti-suburb you are anti-progress.
\section{Pre-Transit Suburbs}
\section{Transit Suburbs}
\section{Automobile Suburbs}
\section{Edge City}
\chapter{Roads And Their Usage}
\begin{quotation}
    Not all roads lead to Rome, but all roads lead to roam.
\end{quotation}
\minitoc
\section{Residential Streets}
\section{Freeways}
\section{Feeders}
\section{Arterials}
\chapter{Malls}
\begin{quotation}
    All right girls, let's go shopping!
\end{quotation}
\minitoc
\section{Strip Malls}
\section{Malls}
\section{Town Centers}
\bibliographystyle{apalike}
\bibliography{bibliography}
\end{document}
